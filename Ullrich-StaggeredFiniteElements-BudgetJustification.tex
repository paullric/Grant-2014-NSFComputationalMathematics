\documentclass[11pt]{article}

\usepackage{amsmath}
\usepackage{graphicx}
\usepackage{multicol}
\usepackage{natbib}
\usepackage{wrapfig}
\usepackage{hyperref}
\usepackage{tabularx}
\usepackage{setspace}
\usepackage{comment}

\oddsidemargin 0cm
\evensidemargin 0cm

\usepackage[margin=1in]{geometry}

\parindent 0cm
\parskip 0.5cm

\usepackage{fancyhdr}
\pagestyle{plain}
%\fancyhf{}
%\fancyhead[L]{AOSS Reference Sheet}
%\fancyhead[CH]{test}
\fancyfoot[C]{Page \thepage}

\newcommand{\vb}{\mathbf}
\newcommand{\diff}[2]{\frac{d #1}{d #2}}
\newcommand{\diffsq}[2]{\frac{d^2 #1}{{d #2}^2}}
\newcommand{\pdiff}[2]{\frac{\partial #1}{\partial #2}}
\newcommand{\pdiffsq}[2]{\frac{\partial^2 #1}{{\partial #2}^2}}
\newcommand{\topic}{\textbf}
\newcommand{\arcsinh}{\mathrm{arcsinh}}
\newcommand{\arccosh}{\mathrm{arccosh}}
\newcommand{\arctanh}{\mathrm{arctanh}}


\begin{document}

{\large \textbf{A New Approach for the Computational Needs of Next-Generation High-Resolution Climate Simulations}}

\textbf{PI:} Dr. Paul Ullrich, University of California, Davis

\textbf{Co-PI:} Dr. Per-Olaf Persson, University of California, Berkeley

\vspace{0.3cm}

{\Large \textbf{Budget Justification}}

\section{Inflation}
\vspace{-0.3cm}

The inflation rate is assumed to be 3\% per year on salaries and travel expenses.

\section{Salaries}
\vspace{-0.3cm}

\textit{Principal Investigator} \\
The principal investigator [Paul Ullrich] will be reimbursed for one month of summer expenses starting in year 1.  This rate amounts to \$7,564 per month in year 1, \$7,790 per month in year 2 and \$8,024 per month in year 3.

\textit{Postdoctoral Scholar} \\
One postdoctoral scholar step (\$54,000 / year) with PhD in mathematics, scientific computing or computer science will be hired for 2 years at 100\% effort on this project, beginning in July 2015.  The postdoc will be primarily in charge of implementation of the Staggered Nodal Finite Element Methods (SNFEM) framework using the Tempest framework as base, development of test problems and additional assistance with analysis of the SNFEM.

\textit{Graduate Student Researcher} \\
One graduate student researcher (GSR4; 9 months at 48\%, 3 summer months at 100\%) will be funded for performing the analysis of the staggered nodal finite element method, to study the problem of the treatment of the horizontal pressure gradient force and horizontal viscosity in the model.  This project will be conducted as part of his/her degree with a thesis option.  The monthly salary rate for a GSR IV is \$3,775 for July 2015 through June 2016, with inflation applied in subsequent years.  This amounts to \$27,633 in year 1, \$28,462 in year 2 and \$29,316 in year 3.

\section{Tuition Expenses}
\vspace{-0.3cm}

Resident tuition expenses for the GSR at UC Davis are \$17,460 in year 1.  This amount is expected to escalate at 10\% per year over the duration of the project, amounting to \$19,205 in year 2 and \$21,126 in year 3.

\section{Fringe Benefits}
\vspace{-0.3cm}

Fringe benefit rates for those working on the project are standard UC Davis rates, as follows:  Faculty summer salary (17.0\% in year 1, 18.0\% in year 2, 18.5\% in year 3), Postdoctoral researcher (17.0\% in Year 1 and 18.0\% in Year 2) and Graduate Student Researcher (1.3\%).

\section{Travel and Living}
\vspace{-0.3cm}

The budget includes domestic travel costs associated with two annual trips to domestic conferences (2016 Dynamical Core Model Intercomparison Project workshop, 2017 PDEs on the Sphere conference and/or 2017 SIAM Computer Science and Engineering conference).  All trips are estimated to require 5 days.  Flights from San Francisco, CA are assessed at \$450 / ticket.  Subsidence, including hotel and per diem food expenses, is estimated at \$300 per day for all destinations.  Conference registration fees are estimated at \$425 each, and ground transportation is assessed at \$60 per day.  The total for domestic travel amounts to \$5,350 in year 1, \$5,511 in year 2 and \$5,676 in year 3.

\section{Other Direct Costs}
\vspace{-0.3cm}

\subsection{Computer Workstation}

A one-time expense of \$3000 will cover a computer workstation for the postdoctoral scholar in year 1 and 2. As stated in the proposal, since the postdoctoral scholar will be working between UC Davis, UC Berkeley, LBNL, and NCAR some sort of mobile computing platform will be needed. Indirect costs can only be used for the purchase of a desktop workstation, which is insufficient for completion of the project. 

\subsection{Software Licenses}

Software license charges include \$138 / year / license for the mathematics software package Maple and one \$165 / year / license for the software package Matlab.  This amounts to \$606 / year in year 1 and 2 and \$303 per year in year 3.  These software packages will be used by the student and postdoc for data processing, intercomparison and modeling.  Other software, such as Microsoft Office, is available for free via a UC Davis license.

\subsection{Publication Costs}

Publication costs are incurred from publication of work produced by this project.  The estimated cost is \$1,000 for year 1 and \$3,000 per year for years 2 and 3.

\subsection{Other Direct Costs: Tuition}

The UC Davis 2014-15 estimated graduate California resident student fees are \$16,541 per year.  Tuition is expected to increase by 10\% for each academic year thereafter.  This amounts to \$17,460, \$19,205 and \$21,126 for each year of the proposal.

\section{Indirect Cost}
\vspace{-0.3cm}

Indirect costs are charged by UC Davis on salaries, supplies, travel and hosting at a rate of 56.5\% in year 1 and 57.0\% in year 2 and 57.0\% in year 3.

\end{document}