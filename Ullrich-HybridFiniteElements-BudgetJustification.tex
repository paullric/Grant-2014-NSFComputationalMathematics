\documentclass[11pt]{article}

\usepackage{amsmath}
\usepackage{graphicx}
\usepackage{multicol}
\usepackage{natbib}
\usepackage{wrapfig}
\usepackage{hyperref}
\usepackage{tabularx}
\usepackage{setspace}
\usepackage{comment}

\oddsidemargin 0cm
\evensidemargin 0cm

\usepackage[margin=1in]{geometry}

\parindent 0cm
\parskip 0.5cm

\usepackage{fancyhdr}
\pagestyle{plain}
%\fancyhf{}
%\fancyhead[L]{AOSS Reference Sheet}
%\fancyhead[CH]{test}
\fancyfoot[C]{Page \thepage}

\newcommand{\vb}{\mathbf}
\newcommand{\diff}[2]{\frac{d #1}{d #2}}
\newcommand{\diffsq}[2]{\frac{d^2 #1}{{d #2}^2}}
\newcommand{\pdiff}[2]{\frac{\partial #1}{\partial #2}}
\newcommand{\pdiffsq}[2]{\frac{\partial^2 #1}{{\partial #2}^2}}
\newcommand{\topic}{\textbf}
\newcommand{\arcsinh}{\mathrm{arcsinh}}
\newcommand{\arccosh}{\mathrm{arccosh}}
\newcommand{\arctanh}{\mathrm{arctanh}}


\begin{document}

{\large \textbf{Addressing The Needs Of Next-Generation High-Resolution Climate Simulations}}

\textbf{Principal Investigator:} Dr. Paul Ullrich, University of California, Davis

{\Large \textbf{Budget Justification}}

\section{Inflation}
\vspace{-0.3cm}

The inflation rate is assumed to be 3\% per year on salaries and travel expenses.

\section{Salaries}
\vspace{-0.3cm}

\textit{Principal Investigator} \\
The principal investigator [Paul Ullrich] will be reimbursed for one month of summer expenses starting in FY15, following the standard UC Davis salary track for fiscal year faculty.  This rate amounts to \$8201 per month in FY15 and \$8437 per month in FY16.

\textit{Postdoctoral Scholar} \\
One postdoctoral scholar step (\$54000 / year) with PhD in Applied Mathematics will be hired for 2 years at 100\% on this project, beginning in July 2014.  The postdoc will be primarily in charge of implementation of the Hybrid Finite Element Methods (HFEM) framework using the Tempest framework as base, development of test problems and additional assistance with analysis of the HFEM.

\textit{Graduate Student Researchers} \\
One graduate student researcher (GSR4; 9 months at 48\%, 3 summer months at 100\%) will be recruited in FY14 for performing the analysis of the hybrid finite element method, verification / validation and to study the problem of the treatment of the horizontal pressure gradient force in the model.  This project will be conducted as part of his/her degree with a thesis option.  The monthly salary rate for a GSR4 is \$3775 for July 2014 through June 2015, with inflation applied in subsequent years.

\section{Fringe Benefits}
\vspace{-0.3cm}

Fringe benefit rates for those working on the project are standard UC Davis rates, as follows:  Faculty summer salary (17.0\% in Year 2, 18.0\% in year 3), Postdoctoral researcher (16.0\% in Year 1 and 17.0\% in Year 2) and Graduate Student Researcher (1.3\%).

\section{Travel and Living}
\vspace{-0.3cm}

The budget includes domestic flight costs to major conferences, including the PDEs on the Sphere conference (Boulder, CO, 2014), the SIAM geosciences conference (2015), the SIAM CS\&E conference (2015) and the Dynamical Core Model Intercomparison Project (DCMIP) summer school (Summer 2016).  Specifically, the budget includes two trips per year for each of the postdoctoral researcher and the graduate student attendee (4 trips in each year).  Flight costs are estimated at \$460 for a flight between Sacramento, CA and Denver, CO (for the PDEs conference and DCMIP summer school) and \$600 for a domestic flight within the continental US.  The domestic subsidence rate is estimated to be \$180 per day (which includes hotel and per diem meal costs) plus local transportation costs of \$30 per day.  Registration fees are estimated at \$600 per year.

\section{Other Direct Costs}
\vspace{-0.3cm}

\subsection{Software Licenses}

Software license charges include \$138 / year / license for the mathematics software package Maple and one \$165 / year / license for the software package Matlab.  These software packages will be used by the student and postdoc for data processing, intercomparison and modeling.  Other software, such as Microsoft Office, is available for free via a UC Davis license.

\subsection{Publication Costs}

Publication costs are incurred from publication of work produced by this project.  The estimated cost is \$1000 per year for publication costs in year 1 and 2 and \$3000 for publication costs in year 3.

\subsection{Other Direct Costs: Tuition}

The UC Davis 2014-15 estimated graduate California resident student fees are \$5905 per quarter, with no tuition paid during the summer quarter.  Tuition is expected to increase by 10\% for each academic year thereafter.  This amounts to \$17,715, \$19,487 and \$21,436 for years 1, 2 and 3.

\section{Indirect Cost}
\vspace{-0.3cm}

Indirect costs are charged by UC Davis on salaries, supplies, travel and hosting at a rate of 55.5\% in Year 1, 56.5\% in Year 2 and 57.0\% in Year 3.

\end{document}