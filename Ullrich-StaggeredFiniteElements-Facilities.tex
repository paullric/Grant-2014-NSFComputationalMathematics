\documentclass[11pt]{article}

\usepackage{amsmath}
\usepackage{graphicx}
\usepackage{multicol}
\usepackage{natbib}
\usepackage{wrapfig}
\usepackage{hyperref}
\usepackage{tabularx}
\usepackage{setspace}
\usepackage{comment}

\oddsidemargin 0cm
\evensidemargin 0cm

\usepackage[margin=1in]{geometry}

\parindent 0cm
\parskip 0.5cm

\usepackage{fancyhdr}
\pagestyle{plain}
%\fancyhf{}
%\fancyhead[L]{AOSS Reference Sheet}
%\fancyhead[CH]{test}
\fancyfoot[C]{Page \thepage}

\newcommand{\vb}{\mathbf}
\newcommand{\diff}[2]{\frac{d #1}{d #2}}
\newcommand{\diffsq}[2]{\frac{d^2 #1}{{d #2}^2}}
\newcommand{\pdiff}[2]{\frac{\partial #1}{\partial #2}}
\newcommand{\pdiffsq}[2]{\frac{\partial^2 #1}{{\partial #2}^2}}
\newcommand{\topic}{\textbf}
\newcommand{\arcsinh}{\mathrm{arcsinh}}
\newcommand{\arccosh}{\mathrm{arccosh}}
\newcommand{\arctanh}{\mathrm{arctanh}}

\begin{document}

\appendix

\setcounter{section}{8}

\section{Facilities, Equipment and Other Resources}

\subsection{Unfunded Collaborations}

This project will enlist the expertise of external collaborators Francis Giraldo and Mark Taylor, both of whom have substantial experience in the development and implementation of finite element methods for atmospheric modeling.  In particular, Francis Giraldo has worked on both continuous and discontinuous finite element formulations, mesh refinement, and stabilization.  Mark Taylor is the lead developer of the Community Atmosphere Model Spectral Element (CAM-SE) dynamical core, which is currently the default dynamical solver for the widely used Community Earth System Model (CESM).  The expertise of both collaborators will be invaluable for this project.  Letters of collaboration are attached.

\subsection{Computational Resources}

Over the course of this project computational resources for software development, testing and data analysis will be required.  Several avenues for obtaining computational resources are available to the PI and co-PI, which will be pursued to ensure the success of this work.

\paragraph{University of California, Davis:} The University of California, Davis has recently invested in building a campus-wide computing cluster for the advancement of scientific research in the environmental sciences, of which a significant portion was provided as part of the PI's start-up package.  Consequently the research team for this project will have high-priority access to this cluster, with over 1,500 processors, including full-time technical support from the campus computing team.  Laptops for the  graduate student researcher have been provided by the PI's start-up package.

\paragraph{NSF and DoE Supercomputing Facilities:}  Through his connections with Lawrence Berkeley Laboratory, the PI has access to computational resources at the National Energy Research Scientific Computing Center (NERSC), specifically on the Department of Energy Hopper supercomputing system and upcoming Cori supercomputing system.  Further, if this project is awarded, the PI will apply for additional computational resources from the Department of Energy and on the Yellowstone supercomputing system.

\end{document}
