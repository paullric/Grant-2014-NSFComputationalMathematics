\documentclass[11pt]{article}

\usepackage{amsmath}
\usepackage{graphicx}
\usepackage{multicol}
\usepackage{natbib}
\usepackage{wrapfig}
\usepackage{hyperref}
\usepackage{tabularx}
\usepackage{setspace}
\usepackage{comment}

\oddsidemargin 0cm
\evensidemargin 0cm

\usepackage[margin=1in]{geometry}

\parindent 0cm
\parskip 0.5cm

\usepackage{fancyhdr}
\pagestyle{plain}
%\fancyhf{}
%\fancyhead[L]{AOSS Reference Sheet}
%\fancyhead[CH]{test}
\fancyfoot[C]{Page \thepage}

\newcommand{\vb}{\mathbf}
\newcommand{\diff}[2]{\frac{d #1}{d #2}}
\newcommand{\diffsq}[2]{\frac{d^2 #1}{{d #2}^2}}
\newcommand{\pdiff}[2]{\frac{\partial #1}{\partial #2}}
\newcommand{\pdiffsq}[2]{\frac{\partial^2 #1}{{\partial #2}^2}}
\newcommand{\topic}{\textbf}
\newcommand{\arcsinh}{\mathrm{arcsinh}}
\newcommand{\arccosh}{\mathrm{arccosh}}
\newcommand{\arctanh}{\mathrm{arctanh}}

\begin{document}

\thispagestyle{empty}

\appendix

\setcounter{section}{9}

\section{Postdoctoral Research Mentoring Plan}

One postdoctoral researcher will be funded on this project for two years.  The PI aims, through advisement and mentoring over their tenure, to give them the greatest opportunity to achieve the goal of meaningful employment in the field of scientific computing and the geosciences.  The PI's close connections with collaborators at the National Center for Atmospheric Research, Lawrence Berkeley National Laboratory and Sandia National Laboratory is expected to be advantageous to the future career of a prospective postdoc.  The postdoctoral researcher's development will also be enhanced through a program of mentoring activities and interaction with key members of the field from both academic and national research institutions.  This mentoring plan will provide the skills, knowledge and experience to prepare the postdoctoral researcher for future excellence in his career.  Specifically, this mentoring plan includes the following elements:
\begin{itemize}
\item The opportunity to engage as both a participant and a guest lecturer in a graduate-level course on global atmospheric model design and implementation (ATM298).  The postdoctoral scholar will also be expected to assist students with model use for course projects, which tackle specific scientific problems of interest to participants.
\item Seminars, workshops and individual consultations on how to identify research funding opportunities and write competitive proposals, offered through the University of California, Davis.
\item Mentoring, encouragement and funding for the preparation of publications for submission to major journals.
\item Opportunity to visit Lawrence Berkeley National Laborotory (LBNL), Sandia National Laboratory (SNL), the Naval Postgraduate School (NPS) and the National Center for Atmospheric Research (NCAR) to network with key senior members of the field, including Dr. Phillip Colella (LBNL), Dr. Mark Taylor (SNL), Dr. Francis Giraldo (NPS) and Dr. Peter Lauritzen (NCAR).  Further opportunities to interact with relevant visiting scholars when they visit the University of California, Davis as part of the PI's broader research program or as part of the school's visiting speaker series.
\item Travel to at least two conferences each year, including the SIAM Geosciences conference, the SIAM conference on Computer Science and Engineering, the conference on Partial Differential Equations on the Sphere and the American Geophysical Union Fall Meeting.
\item Opportunity for engagement of the postdoctoral researcher in undergraduate-level course instruction as a lecturer in the University of California, Davis department of mathematics.
\item Support for the involvement of the postdoc in postdoctoral associations to gain experience in leadership, teamwork, event planning and project management.  At UC Davis, postdoctoral scholars have access to the UC Davis Postdoctoral Scholars Association (PSA), the UC Council of Postdoctoral Scholars and the National Postdoctoral Association.
\end{itemize}

Success of this mentoring plan will be assessed via regular weekly meetings with the postdoctoral researcher.  The PI aims to actively track the postdoctoral researcher's progress towards his career goals both for the duration of the research position, and following his successful completion of the postdoc.

\end{document}
