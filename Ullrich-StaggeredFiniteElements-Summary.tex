\documentclass[11pt]{article}

\usepackage{amsmath}
\usepackage{graphicx}
\usepackage{multicol}
\usepackage{natbib}
\usepackage{wrapfig}
\usepackage{hyperref}
\usepackage{tabularx}
\usepackage{setspace}

\oddsidemargin 0cm
\evensidemargin 0cm

\usepackage[margin=1in]{geometry}

\parindent 0cm
\parskip 0.5cm

\usepackage{fancyhdr}
\pagestyle{plain}
%\fancyhf{}
%\fancyhead[L]{AOSS Reference Sheet}
%\fancyhead[CH]{test}
\fancyfoot[C]{Page \thepage}

\newcommand{\vb}{\mathbf}
\newcommand{\diff}[2]{\frac{d #1}{d #2}}
\newcommand{\diffsq}[2]{\frac{d^2 #1}{{d #2}^2}}
\newcommand{\pdiff}[2]{\frac{\partial #1}{\partial #2}}
\newcommand{\pdiffsq}[2]{\frac{\partial^2 #1}{{\partial #2}^2}}
\newcommand{\topic}{\textbf}
\newcommand{\arcsinh}{\mathrm{arcsinh}}
\newcommand{\arccosh}{\mathrm{arccosh}}
\newcommand{\arctanh}{\mathrm{arctanh}}

\begin{document}

\appendix

\addtocounter{section}{1}

\section{Project Summary}
\vspace{-0.2cm}

The next century is expected to see unprecedented changes to the climate system with extensive socioeconomic repercussions.  Since large-scale intervention and experimentation on the Earth system is impossible, global atmospheric modeling systems have been built to provide a laboratory for understanding future climate change.  However, restrictions on computational cost have inherently limited the finest resolution of our uniform-resolution global atmospheric models to 50-100 kilometers, far beyond the scale necessary for capturing many facets of regional climate.  Dynamically downscaled simulations that use global climate model data for driving the boundary conditions of regional-scale models are also known to suffer from spurious boundary noise due to inconsistencies in the regional and global numerics.  Consequently, there is a pressing need to develop a next generation of global atmospheric models that can reach high spatial resolutions and local scales.

This proposal aims to develop a new robust and accurate general circulation model that provides a nearly optimal treatment of wave-like motion, is scalable on large-scale parallel systems, remains valid at all grid resolutions, improves on existing treatments of topography and supports both static and adaptive mesh refinement.  To support this goal, this proposal will focus on analysis and implementation of Staggered Nodal Finite Element Methods (SNFEMs).  These methods incorporate the scalability and high-order accuracy of finite element methods with the greatly improved treatment of wave-like motion associated with staggered methods.  Both a horizontal and vertical discretization will be developed, and the method implemented in the Tempest atmospheric modeling framework.  This approach is expected to improve on topographically-driven pressure gradient errors, and will be coupled with a new stratification-aware viscosity for stabilization.  To reach the grid resolutions needed for regional simulations, a static and adaptive mesh refinement treatment will be implemented using block-structured refinement.

\vspace{-0.5cm}
\subsection*{Intellectual Merit}
\vspace{-0.5cm}

The issues addressed in this proposal are key for pushing atmospheric models to the resolutions needed to answer pertinent questions about climate change on the regional scale.  With an improved treatment of fine-scale motions, scalability to large scale parallel systems, improvements in topographic representation and support for mesh refinement, this work is potentially transformative on the way that atmospheric models are constructed.  Further, the numerical methods developed as part of this research have the potential for being applied to many important problems in computational fluid dynamics, ranging from aerospace to computational biology.

\vspace{-0.5cm}
\subsection*{Broader Impacts}
\vspace{-0.5cm}

The goal of this project is to improve the state-of-the-art for global climate prediction and weather forecasting by tackling a number of outstanding problems with these models.  This work has substantial societal impact ranging from weather forecasting, water resource management, agricultural planning and urban development.  The improved treatment of topography addressed by this proposal will make climate models more applicable to mountainous regions and will provide a test bed for the development of non-hydrostatic and multi-resolution parameterizations of sub-grid-scale processes.  This project will build an interdisciplinary partnership between atmospheric science and computational science by a mutual exchange of expertise.  This research will also be a driver for the 2016 Dynamical Core Model Intercomparison Project (DCMIP) workshop, which will address the topic of multi-resolution modeling.

\end{document}
