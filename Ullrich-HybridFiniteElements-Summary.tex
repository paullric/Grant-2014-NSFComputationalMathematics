\documentclass[11pt]{article}

\usepackage{amsmath}
\usepackage{graphicx}
\usepackage{multicol}
\usepackage{natbib}
\usepackage{wrapfig}
\usepackage{hyperref}
\usepackage{tabularx}
\usepackage{setspace}

\oddsidemargin 0cm
\evensidemargin 0cm

\usepackage[margin=1in]{geometry}

\parindent 0cm
\parskip 0.5cm

\usepackage{fancyhdr}
\pagestyle{plain}
%\fancyhf{}
%\fancyhead[L]{AOSS Reference Sheet}
%\fancyhead[CH]{test}
\fancyfoot[C]{Page \thepage}

\newcommand{\vb}{\mathbf}
\newcommand{\diff}[2]{\frac{d #1}{d #2}}
\newcommand{\diffsq}[2]{\frac{d^2 #1}{{d #2}^2}}
\newcommand{\pdiff}[2]{\frac{\partial #1}{\partial #2}}
\newcommand{\pdiffsq}[2]{\frac{\partial^2 #1}{{\partial #2}^2}}
\newcommand{\topic}{\textbf}
\newcommand{\arcsinh}{\mathrm{arcsinh}}
\newcommand{\arccosh}{\mathrm{arccosh}}
\newcommand{\arctanh}{\mathrm{arctanh}}

\begin{document}

\appendix

\addtocounter{section}{1}

\section{Project Summary}
\vspace{-0.2cm}

The next century will see unprecedented changes to the climate system with extensive socioeconomic repercussions.  Global Earth-system models, which incorporate the sum total of our knowledge of the climate system, have been the primary tool for understanding these changes.  However, the disparity between global and regional scales has made it difficult to diagnose changes to regional climate which occur due to shifts in the whole Earth system:  Restrictions on computational cost have inherently limited the finest resolution of our uniform-resolution climate models to 50-100 kilometers, far beyond the scale necessary for capturing many facets of regional climate.  Consequently, the development of accurate regional climate simulations at horizontal resolutions at or below 10 km is among the top short-term imperatives for the climate modeling community.

This proposal addresses a number of key issues that are required for accurate simulation at fine scales.  First, it examines a new class of Hybrid Finite Element Methods (HFEM) which are scalable on massively parallel systems, portable to hybrid architectures, and further have the potential for improving the accuracy of both the horizontal and vertical discretizations.  Second, it addresses the need for methods which are effective at simulating the unapproximated non-hydrostatic equations of motion in the atmosphere to high accuracy and accounting for the high aspect ratio between horizontal and vertical motion ($\sim 100$).  Third, it tackles the issue of accurately computing pressure gradient forces over highly variable and steep topography, one of the major outstanding issues for atmospheric models.  Fourth, it proposes that these topics be addressed within a variable resolution framework, which supports concurrent simulation at multiple refinement levels, with the goal of testing scale-aware physical parameterizations.

\vspace{-0.5cm}
\subsection*{Intellectual Merit}
\vspace{-0.5cm}

The issues addressed in this proposal are key for pushing atmospheric models to the resolutions needed to answer pertinent questions about climate change on the regional scale.  With an improved treatment of fine-scale motions, scalability to large scale parallel systems, resolution of the issue of sharply varying topography in atmospheric models and support for multi-resolution this work is potentially transformative on the way that atmospheric models are constructed.  To give just one example, an understanding of declining western snow pack requires high resolution and accurate treatment of sharply varying topography.  Further, the numerical methods developed as part of this research have the potential for being applied to many important problems in computational fluid dynamics, ranging from aerospace to computational biology.

\vspace{-0.5cm}
\subsection*{Broader Impacts}
\vspace{-0.5cm}

The goal of this project is to improve the state-of-the-art for global climate and weather forecasting models by improving the treatment of the underlying dynamics within these models.  This work has substantial societal impact ranging from weather forecasting, water resource management, agricultural planning and urban development.  The improved treatment of topography addressed by this proposal will make climate models more applicable to regions of sharply varying topography, such as around California's central valley.  Further, this work will make it easier to develop scale-aware physical parameterizations.  This project will also build a partnership between the National Center for Atmospheric Research (NCAR) and UC Davis by a mutual exchange of expertise on atmospheric dynamics.  This research will be a driver for the 2016 Dynamical Core Model Intercomparison Project (DCMIP) workshop, which will address the topic of multi-resolution modeling.  Historically this workshop has brought together over a dozen international modeling groups and over 30 graduate students in a mutual exchange of knowledge.

\end{document}
